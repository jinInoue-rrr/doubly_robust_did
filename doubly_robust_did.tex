\documentclass{beamer}
\usepackage{luatexja} 
\usepackage[ipaex]{luatexja-preset} 
\renewcommand{\kanjifamilydefault}{\gtdefault} 

\usepackage{bm} 
\usepackage{mathtools} 
\usepackage{amsmath} 
\usepackage{amsthm} 

\newtheorem{thm}{Theorem}[section] 
\newtheorem{lem}[thm]{Lemma} 
\newtheorem{prop}[thm]{Proposition} 
\newtheorem{assumption}[thm]{Assumption} 
\usetheme{Madrid} 
\setbeamertemplate{navigation symbols}{} 

\title{Introduction to Doubly Robust Differece-in-Differences} 
\author{M1 Inoue Jin}
\institute{Hitotsubashi University} 

\begin{document}

\begin{frame}
    \titlepage
\end{frame}

\begin{frame}\frametitle{Doubly Robust DiD}
    \begin{itemize}
        \item Doubly Robust DiD was proposed by Sant'Anna and Zhao(2020)
        \item This paper suggests doubly robust estimators for the ATT in DiD research designs.
        \item Authors also state that one can construct doubly robust DiD estimators for the ATT that are also doubly robust for inference.
    \end{itemize}
\end{frame}

\begin{frame}\frametitle{Notations}
    \begin{itemize}
        \item $Y_{it}$ : the outcome of interest for unit $i$ at time $t$
        \item $t$ : researchers have access to outcome data in a pre-treated period $t = 0$ and in a post-treatment period $t = 1$.
    \end{itemize}
\end{frame}

\begin{frame}\frametitle{The parameter of interest}
    \begin{itemize}
        \item nanka iikanjino bunsyou for the introduction of ATT
        \item Difference in Differences identify the average treatment effect for the treated (ATT). 
    \end{itemize}
    \begin{block}{The parameter of interest:ATT}
        $\tau = \mathbb{E}[Y_{i1}(1) - Y_{i1}(0)| D_{i} = 1] = \mathbb{E}[Y_{1} |D = 1] - \mathbb{E}[Y_{1}(0) | D = 1]$
    %Y_{i1}(1) = Y_{i1} if D_{i} = 1
    \end{block}
\end{frame}

\begin{frame}\frametitle{Assumption 1.}
    \begin{assumption}[Random Sampling]
        the data $\{Y_{i0}, Y_{i1}, D_{i}, X_{i}\}^{n}_{i = 1}$ are independent and identically distributed (iid)
    \end{assumption}
\end{frame}

\begin{frame}\frametitle{Assumption 2.}
    \begin{assumption}[Conditional PTA]
        $\mathbb{E}[Y_{1}(0) - Y_{0}(0)| D = 1, X] = \mathbb{E}[Y_{1}(0) - Y_{0}(0)|D = 0, X]$ almost surely (a.s.).
    \end{assumption}
    \begin{itemize}
        \item This assumption is called "Conditional Parallel Trend Assumption"
        \item It is crucial for the most of DiD literature.
    \end{itemize}
\end{frame}

\begin{frame}\frametitle{Assumption 3.}
    \begin{assumption}[Overlap Condition]
        For some $\varepsilon > 0$, $\mathbb{P}(D = 1) > \varepsilon$ and $\mathbb{P}(D = 1|X) \leq 1 - \varepsilon$ a.s.
    \end{assumption}
    \begin{itemize}
        \item This assumption states that at least a small fraction of the population is treated and that for every value of the covariates X, there is at least a small probability that the unit is not treated.%IPWの部分が発散してしまう場合を排除するための仮定
        \item These assumptions are standard in conditional DID methods.
    \end{itemize}
\end{frame}

\begin{frame}\frametitle{Doubly Robust DiD estimand:Notation}
    \begin{itemize}
        \item $\pi(X)$ : an arbitrary model for the true, unknown propensity score
        \item $\Delta Y = Y_{1} - Y_{0}$:the difference of observed outcomes
        \item $\mu ^{p}_{d,\Delta} \equiv \mu ^{p}_{d,1}(X) - \mu ^{p}_{d,0}(X)$
        \begin{itemize}
            \item $\mu ^{p}_{d,t}$ : a model for the true, unknown outcome regression $m^{p}_{d,t} \equiv \mathbb{E}[Y_{t}|D = d, X = x]],d,t = 0,1$
        \end{itemize}
    \end{itemize}
\end{frame}

\begin{frame}\frametitle{Doubly Robust DiD estimand}
    \begin{block}{DR DID estimand when panel data are available}
        \begin{equation*}
        \tau^{dr,p} = \mathbb{E}\lbrack (w^{p}_{1}(D)- w^{p}_{0}(D,X;\pi))(\Delta Y - \mu ^{p}_{0,\Delta}(X)))\rbrack
        \end{equation*}
        where, for a generic $g$,
        \begin{equation*}
            w^{p}_{1}(D) = \frac{D}{ \mathbb{E}[D]},w^{p}_{0}(D,X;g) = \frac{ \frac{g(X)(1-D)}{1-g(X)}}{ \mathbb{E}[\frac{g(X)(1-D)}{1-g(X)}]}
        \end{equation*}
    \end{block}
\end{frame}

\begin{frame}\frametitle{Doubly Robust DiD estimand}
    \begin{block}{Theorem 1.}
        Let Assumptions 1-3 hold. When panel data are available, $\tau ^{dr,p} = \tau$ if either (but not necessarily both) $\pi(X) = p(X)$ a.s. or $\mu ^{p}_{\Delta}(X) = m^{p}_{0,1}(X) - m^{p}_{0,0}(X)$
    \end{block}

    \begin{itemize}
        \item This theorem states that at least one of the outcome regression model or propensity score model is correctly specified, we can recover the average treatment effect for the treated(ATT).
        \begin{itemize}
            \item Case1:the propensity score model is correctly specified($\pi(X) = p(X) = P(D = 1|X)$), but the outcome regression model is miss-specified.
            \item Case2:the outcome regression model is correcly specified($\mu ^{p}_{0,\Delta}(X) = m^{p}_{0,\Delta}(X) = \mathbb{E}[Y_{1}|D = 0, X] - \mathbb{E}[Y_{0}|D = 0, X]$), but the propensity score model is miss-specified.
        \end{itemize}
    \end{itemize}

\end{frame}

\begin{frame}\frametitle{Proof of Theorem 1.(Case1)}
    In Case 1, the propensity score model $\pi(X)$ is equivalent to the true propensity score $p(X) = P(D = 1|X)$. 

    First, calculate the difference of two weights $w^{p}_{1}(D) - w^{p}_{0}(D)$:
    \begin{eqnarray*}
        w^{p}_{1}(D) - w^{p}_{0}(D) &=& \frac{D}{ \mathbb{E}[D]} - \frac{ \frac{\pi(X)(1-D)}{1-\pi(X)}}{ \mathbb{E}[\frac{\pi(X)(1-D)}{1-\pi(X)}]} \\
        &=& \frac{D}{\mathbb{E}[D]}
    \end{eqnarray*}
\end{frame}


\end{document}